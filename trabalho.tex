\documentclass[12pt]{article}

\usepackage{graphicx}
\usepackage{mathpazo}
\usepackage[utf8]{inputenc}
\usepackage[T1]{fontenc}
\usepackage{microtype}
\usepackage[brazil]{babel}
\usepackage{csquotes}
\usepackage[backend=biber, style=numeric, sorting=none, sortcites=true]{biblatex} % Usando biber como backend
\usepackage{bm}
\usepackage{amsfonts}
\usepackage{amsmath}
\usepackage{indentfirst}
\usepackage{float}
\usepackage{amsmath} % Para \text

\addbibresource{referencias.bib}

\title{Reconstrução de Imagens via Métodos Bayesianos}
\author{Gabriel Ligabô}

\begin{document}

\maketitle

\section{Introdução}

A reconstrução de imagens é um problema clássico que busca recuperar uma imagem original a partir de uma versão degradada. Esse tipo de problema é importante em diversos campos de estudo, um exemplo de sua utilidade é na reconstrução de imagens de tomografia computadorizada \cite{kas88}. Este trabalho explora uma abordagem bayesiana para a deconvolução e redução de ruído de imagens, focando em um método que combina as vantagens das representações nos domínios da frequência e wavelet.

\section{Metodologia}

\subsection{Definição do Problema}

Queremos restaurar a imagem original $x \in \mathbb{R}^n$ a partir de uma observação degradada $y \in \mathbb{R}^m$, relacionada por um operador linear $H \in \mathbb{R}^{m \times n}$. Em geral, assume-se que $m = n = N$.

A formulação matemática do problema é:
\[
y = Hx + n,
\]
onde $n$ representa ruído aditivo gaussiano com média zero e variância $\sigma^2$. Isto é, assumimos que:
\begin{align*}
n &\sim \mathcal{N}(0, \sigma^2 I). \\
p(n) &= \mathcal{N}(n|0, \sigma^2 I).
\end{align*}

Neste trabalho, focamos em problemas onde H modela uma convolução periódica (e.g. os pixels na borda direita são tratados como vizinhos dos pixels na borda esquerda) e espacialmente invariante ("borrão" é o mesmo em toda a imagem). Nesse caso, a matriz H é do tipo bloco-circulante e, por essa propriedade, pode ser diagonalizada pela Transformada Discreta de Fourier (DFT) bidimensional:

\[
H = U^HDU
\]

Isto é, estamos diagonalizando o operador de convolução H usando a Transformada Discreta de Fourier Bidimensional (2D DFT). Dessa forma, $U$ é a matriz que representa a transformada, $U^H$ é a transposta conjugada e também a inversa de $U$, e por fim $D$ é uma matriz diagonal que contém os coeficientes DFT do operador de convolução representado por H.


\subsection{Abordagem no Domínio da Frequência (FFT)}


As definições feitas na seção anterior implicam que a multiplicação $Hx$ pode ser feita no domínio de Fourier:
\[
Hx = U^HDUx
\]

Se H for inversível, então a estimativa de x é facilmente obtida ao fazer:
\[
\hat{x} = U^HD^{-1}Uy
\]

No entanto, na maioria dos casos, H não é inversível ou é dita uma matriz mal condicionada \cite{ozturk2000ill} (i.e., a diagonal de D tem valores muito pequenos), o que leva à amplificação do ruído na estimativa anterior. Para contornar esse problema, utiliza-se um método de regularização. Uma escolha comum é o estimador de máxima verossimilhança penalizado (MPLE), ou, sob uma perspectiva Bayesiana, o estimador máximo a posteriori (MAP):
\[
\hat{x} = \arg \max_{x} \{\log{p(y|x)} - pen(x)\}
\]
onde $p(y|x) = \mathcal{N}(y|Hx, \sigma^2I)$ é a função de verossimilhança e $pen(x)$ é uma função de penalidade (ou -log da priori, $p(x)$). Se a priori sobre a imagem $p(x)$ for assumida como uma gaussiana com média $\mu$ e matriz de covariância $G$, a estimativa MAP pode ser escrita como:

\begin{align}
\hat{x} &= \arg \max_{x} \{\frac{-1}{\sigma^2}||Hx - y||^2 - (x - \mu)^HG^{-1}(x-\mu)  \}\\
        &= \mu + GH^H (\sigma^2I + HGH^H)^{-1} (y-H\mu)   
\end{align}

Para que a solução MAP na equação (2) seja computacionalmente eficiente, assume-se que a imagem original $x$ pode ser modelada como um campo gaussiano estacionário com condições de contorno periódicas. Essa premissa implica que a matriz de covariância da priori, $G$, também é bloco-circulante e, portanto, também é diagonalizada pela DFT. Podemos escrever $G = U^H C U$, onde C é uma matriz diagonal.

Com isso, a equação (2) pode ser implementada de forma eficiente no domínio da frequência, resultando no conhecido filtro de Wiener:
\[
\hat{x} = \mu + U^H C D^H (\sigma^2I + DCD^H)^{-1} (Uy - DU\mu)
\]

Apesar da eficiência, essa abordagem tem uma limitação fundamental: imagens do mundo real não são bem modeladas por campos gaussianos estacionários. A energia do sinal de uma imagem típica não se concentra no domínio de Fourier, o que torna difícil separar o sinal do ruído de forma eficaz. Essa limitação motiva a busca por representações mais adequadas, como o domínio Wavelet.


\subsection{Abordagem no Domínio Wavelet (DWT)}
A grande vantagem das transformadas wavelet é que elas fornecem representações esparsas para imagens, ou seja, a maior parte da energia da imagem é capturada por um pequeno número de coeficientes grandes, enquanto a maioria dos coeficientes é próxima de zero.

Nesta abordagem, a imagem $x$ é re-expressa em termos de seus coeficientes wavelet $\theta$ através da Transformada Wavelet Discreta (DWT) Inversa, denotada por $W$:
\[
x = W\theta
\]
O critério de estimação MAP é então formulado no domínio wavelet:
\[
\hat{\theta} = \arg \max_{\theta} \{ \log{p(y|\theta)} - pen(\theta) \} = \arg \max_{\theta} \{ -\frac{||y - HW\theta||^2}{2\sigma^2} - pen(\theta) \}
\]
A função de penalidade $pen(\theta)$ é escolhida para induzir a esparsidade dos coeficientes $\theta$.

No entanto, essa formulação apresenta um grande desafio computacional. O operador combinado $HW$ não é, em geral, bloco-circulante, o que impede a diagonalização eficiente via FFT que era possível com $H$ isoladamente. Além disso, $HW$ não é uma matriz ortogonal, o que descarta o uso de regras de limiarização (thresholding) coeficiente a coeficiente, que são muito eficientes em problemas de remoção de ruído (denoising).


\subsection{O Algoritmo Expectation-Maximization (EM)}
O problema que queremos resolver, após entendermos os conceitos das seções anteriores, é:

\[
y = HW\theta + n
\]

No entanto, como já foi visto na seção anterior, esse tipo de formulação traz diversos imbróglios.

Para superar as dificuldades da abordagem wavelet, Figueiredo e Nowak \cite{1217267} propõem um algoritmo de Expectation-Maximization (EM). A chave do método é introduzir uma variável latente $z$ que desacopla a deconvolução da remoção de ruído (denoising).

O modelo anterior é decomposto em duas etapas:
\[
\begin{cases}
z = W\theta + \alpha n_1 \\
y = Hz + n_2
\end{cases}
\]
onde $n_1 \sim \mathcal{N}(0, I)$ e $n_2 \sim \mathcal{N}(0, \sigma^2I - \alpha^2HH^T)$ são ruídos gaussianos independentes. O parâmetro $\alpha$ é escolhido tal que a matriz de covariância de $n_2$ seja semipositiva definida. Note que se tivéssemos acesso a $z$, o problema seria simplesmente remover o ruído $\alpha n_1$ de $z$ para estimar $W\theta$, um problema de denoising padrão.

O algoritmo EM itera entre dois passos para encontrar a estimativa $\hat{\theta}$:

\begin{itemize}
    \item \textbf{E-step (Expectation):} Calcula a esperança condicional do dado completo, o que, neste caso, se resume a estimar a variável latente $z$ dado a observação $y$ e a estimativa atual da imagem, $\hat{x}^{(t)} = W\hat{\theta}^{(t)}$. A atualização é dada por:
    \[
    \hat{z}^{(t)} = E[z|y, \hat{\theta}^{(t)}] = \hat{x}^{(t)} + \frac{\alpha^2}{\sigma^2} H^T(y - H\hat{x}^{(t)})
    \]
    Este passo é computacionalmente eficiente, pois envolve a matriz $H$ e sua transposta $H^T$, que podem ser implementadas via FFTs no domínio da frequência.
    
    \item \textbf{M-step (Maximization):} Atualiza a estimativa dos coeficientes wavelet, $\hat{\theta}^{(t+1)}$. Neste trabalho, utilizaremos a \textbf{penalização $l_1$}, que é amplamente usada por sua capacidade de induzir esparsidade. A função de penalidade é dada por:
    \[
    pen(\theta) = \tau||\theta||_1 = \tau \sum_i |\theta_i|
    \]
    onde $\tau$ é um parâmetro de regularização positivo que controla a intensidade da esparsidade. O passo de maximização se torna:
    \[
    \hat{\theta}^{(t+1)} = \arg \max_{\theta} \left\{ - \frac{||W\theta - \hat{z}^{(t)}||^2}{2\alpha^2} - \tau \sum_i |\theta_i| \right\}
    \]
    Como a matriz wavelet $W$ é ortogonal ($||W\theta - \hat{z}^{(t)}||^2 = ||\theta - W^T\hat{z}^{(t)}||^2$), o problema se desacopla e pode ser resolvido coeficiente a coeficiente. A solução é a conhecida função \textit{soft-thresholding}. Especificamente, cada componente $\hat{\theta}_i^{(t+1)}$ é obtido aplicando-se a seguinte regra aos coeficientes $\hat{\omega}^{(t)} = W^T\hat{z}^{(t)}$:
    \[
    \hat{\theta}_{i}^{(t+1)} = \text{sgn}(\hat{\omega}_{i}^{(t)}) (|\hat{\omega}_{i}^{(t)}| - \tau\alpha^2)_{+}
    \]
    onde $(\cdot)_+$ denota o operador da parte positiva, i.e., $(x)_+ = \max\{x, 0\}$, e o valor $\tau\alpha^2$ atua como o limiar (threshold) de decisão.
\end{itemize}

\subsection{Esquema Geral do Algoritmo}
O algoritmo combina o melhor dos dois mundos: a diagonalização do operador de convolução no domínio de Fourier e a modelagem esparsa de imagens no domínio wavelet. Cada iteração do algoritmo pode ser resumida da seguinte forma:
\[
\hat{x}^{(t+1)} = \mathcal{P}\left(\hat{x}^{(t)} + \frac{\alpha^2}{\sigma^2}H^T(y - H\hat{x}^{(t)})\right)
\]
onde $\mathcal{P}(\cdot)$ denota a operação de denoising no domínio wavelet: $\mathcal{P}(v) = W \mathcal{D}(W^T v)$, sendo $\mathcal{D}$ a função de encolhimento (shrinkage) aplicada aos coeficientes.

O processo é o seguinte:
\begin{enumerate}
    \item \textbf{Inicialização:} Comece com uma estimativa inicial da imagem, $\hat{x}^{(0)}$, por exemplo, a imagem observada $y$ ou uma estimativa do filtro de Wiener.
    \item \textbf{Iteração $t+1$:}
    \begin{enumerate}
        \item \textbf{(E-step)} Calcule o resíduo $y - H\hat{x}^{(t)}$, aplique o operador $H^T$ e adicione à estimativa atual $\hat{x}^{(t)}$ para obter $\hat{z}^{(t)}$. Esta etapa é realizada eficientemente com FFTs.
        \item \textbf{(M-step)} Aplique a transformada wavelet (DWT) em $\hat{z}^{(t)}$, aplique a função de \textit{soft-thresholding} nos coeficientes e, em seguida, aplique a transformada inversa (IDWT) para obter a nova estimativa da imagem, $\hat{x}^{(t+1)}$.
    \end{enumerate}
    \item \textbf{Convergência:} Repita o passo 2 até que um critério de parada seja satisfeito, como a mudança relativa entre estimativas consecutivas ser menor que um limiar.
\end{enumerate}

\section{Resultados}

Para avaliar o desempenho do algoritmo EM-Wavelet, realizamos um experimento numérico utilizando a imagem de teste "cameraman". A imagem original foi degradada por um borrão de movimento uniforme, modelado por um kernel de convolução de 9x9, e pela adição de ruído gaussiano branco, resultando em uma imagem observada com uma Relação Sinal-Ruído de Pico (PSNR) de 23.19 dB.

O algoritmo foi inicializado com uma estimativa gerada pelo filtro de Wiener (ver Apêndice \ref{ap:wiener} para uma explicação detalhada), que por si só já elevou o PSNR para 26.40 dB. Em seguida, o processo iterativo do EM foi executado. Os hiperparâmetros utilizados foram a wavelet 'haar' com 4 níveis de decomposição e um parâmetro de regularização $\lambda=1$.

\begin{figure}[H]
    \centering
    \includegraphics[width=1\linewidth]{resultados.png}
    \caption{Imagem original, borrada e ruidosa e reconstruída.}
    \label{fig:resultados}
\end{figure}

O método demonstrou uma melhora progressiva a cada iteração, atingindo o melhor resultado na 25ª iteração. A imagem restaurada final alcançou um PSNR de 29.89 dB. Isso representa uma Melhoria na Relação Sinal-Ruído (SNRI) de 6.70 dB em relação à imagem degradada. A Figura \ref{fig:resultados} ilustra a qualidade visual da imagem original, da versão degradada e do resultado da restauração.
\section{Conclusão}

O algoritmo EM para deconvolução de imagens no domínio wavelet provou ser uma abordagem poderosa e computacionalmente viável. Ele combina a eficiência da FFT para lidar com a convolução e a capacidade das wavelets de modelar esparsamente o conteúdo de imagens. Os resultados experimentais, que mostraram um ganho de 6.70 dB em SNRI, demonstram a eficácia do método. A estrutura iterativa, que alterna entre um passo de filtragem no domínio da frequência e um passo de denoising no domínio wavelet, converge para uma solução ótima sob condições suaves, oferecendo resultados de alta qualidade na tarefa de restauração de imagens.
\printbibliography

\appendix
\section{Filtro de Wiener}\label{ap:wiener}

O Filtro de Wiener é um método clássico e computacionalmente eficiente para restaurar uma imagem que foi degradada por borrão e ruído. Embora o algoritmo principal deste trabalho utilize uma abordagem iterativa mais sofisticada, o Filtro de Wiener serve como um excelente ponto de partida.

\subsection{Ideia Central}

Imagine que uma imagem é borrada e ruidosa. Uma abordagem ingênua seria tentar reverter o borrão de forma exata. No entanto, essa "filtragem inversa" tem um efeito colateral grave: ela amplifica o ruído que já existe na imagem, muitas vezes tornando o resultado final pior do que a imagem degradada.

O Filtro de Wiener resolve este problema de forma mais inteligente. Ele não busca apenas reverter o borrão, mas sim encontrar um equilíbrio ótimo entre a nitidez da imagem e a supressão do ruído. O seu objetivo não é ser um filtro inverso perfeito, mas sim produzir a estimativa mais "limpa" possível, no sentido de minimizar o erro quadrático médio entre a imagem estimada e a original.

\subsection{Estratégia: Equilíbrio no Domínio da Frequência}

A implementação do filtro é mais intuitiva no domínio da frequência, para onde a imagem é levada usando a Transformada de Fourier. Neste domínio, a imagem é decomposta em suas frequências constituintes.

Para cada frequência, o filtro avalia a \textbf{Relação Sinal-Ruído (SNR)}, que é a proporção entre a energia do sinal da imagem original e a energia do ruído. Com base nessa avaliação, ele decide como agir:

\begin{itemize}
    \item \textbf{Em frequências com SNR alta:} Onde o sinal da imagem é muito mais forte que o ruído, o filtro age de forma agressiva para reverter o borrão, pois há pouco risco de amplificar o ruído.
    \item \textbf{Em frequências com SNR baixa:} Onde o ruído domina o sinal, o filtro atenua ou até mesmo zera essa frequência. Isso evita que o ruído seja amplificado, mesmo que signifique não reverter completamente o borrão naquela parte específica da imagem.
\end{itemize}

Essencialmente, o filtro faz uma troca inteligente: sacrifica um pouco da nitidez nas áreas ruidosas para obter uma imagem globalmente mais limpa.

\subsection{Aplicação Prática e Papel no Algoritmo}

Na prática, não conhecemos a energia exata da imagem original para calcular a SNR perfeitamente. Por isso, como visto no código deste trabalho, utilizamos uma estimativa para a potência do sinal e a variância conhecida do ruído para criar um \textit{termo de regularização}. Esse termo controla o equilíbrio do filtro.

Devido à sua rapidez (requer apenas algumas operações de FFT) e à sua capacidade de fornecer uma estimativa inicial de boa qualidade, o Filtro de Wiener é frequentemente usado como o ponto de partida para algoritmos iterativos mais complexos. Neste trabalho, ele gera a estimativa $\hat{x}^{(0)}$, que é então refinada progressivamente pelos passos E e M do algoritmo principal para alcançar um resultado superior.\end{document}